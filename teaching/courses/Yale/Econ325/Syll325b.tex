
\documentclass[11pt,pdftex]{article}
%%%%%%%%%%%%%%%%%%%%%%%%%%%%
\usepackage{times}
\usepackage[pdftex]{graphicx}
\usepackage{epsfig}
\usepackage{calc,fancybox,url}
\usepackage{makeidx}
\usepackage[hyperindex,backref]{hyperref}
\usepackage{varioref}

\setlength{\textwidth}{6.5in}
\setlength{\textheight}{8.5in}
\setlength{\topmargin}{-.5in}
\setlength{\oddsidemargin}{.25in}
\setlength{\evensidemargin}{.25in}
\makeindex
\input tcilatex
\QQQ{Language}{
American English
}

\begin{document}

\begin{centering}
{\Large {\bf Econ325b: Economics of Developing Economies }}
\end{centering}
\medskip

\noindent  Professor Lakshmi K. Raut

\noindent Spring 2000, \emph{T,Th:11:30-12:45, Room: ML 104}

\noindent  Office hours Wednesdays 12:00PM-2:00PM, Room \#314 at 28
Hillhouse Ave.

\noindent  E-mail: raut.lakshmi@yale.edu

\noindent \textbf{EXAMS AND GRADING}:

There will be a few regular homework assignments which will carry 20\%, an
empirical homework 25\%, mid-term exam 25\% and the final exam 30\% of your
grade. There will be no make-up exam under any circumstances. The course
materials will be mostly developed in the class. So it is important to
attend all the classes. If you miss any class, please arrange with a
classmate for the lecture notes. Partial lecture notes will be posted on the
internet site for this course.

The announcements, homework assignments, and partial answer keys to
assignments and partial lecture notes will be kept in the following internet
site:

http://pantheon.yale.edu/$\sim $rkl4/Courses/Econ325b.html

\emph{(some of the course materials will require a password which will be
given to you in the class)}.

\noindent \textbf{MID-TERM EXAMS:} the First Mid-Term: ????

\noindent\ \textbf{FINAL EXAM:} As scheduled by the University.

\noindent \textbf{Recommended Texts:}

\begin{itemize}

\item  Charles Jones, ''Introduction to Economic Growth,'' W.W.Norton \&
Company, New York, 1998.

\item  Wilfred Ethier, ''Modern International Economics,'' W.W. Norton, 3rd
Edition, New York,1997.

\item There will be a few chapters of other books, such as (1) Michael Todaro,
"Economic Development'' (Fifth Edition) 1994;(2) Debraj Ray, ''Development
Economics,'' Princeton University Press, New Jersey,1998. These
chapters will be available at the soft reserve of CCL and SSL libraries
for you to consult.

\end{itemize}

\textbf{PART I: OVERVIEW}

\begin{description}
\item  \textbf{1. Introduction: Nature of Development problems(Todaro
Chapter 2, and classnotes)}

\begin{itemize}
\item  Structures of Developing economies

\item  Measurement of living standards

\item  Other indicators
\end{itemize}

\item  \textbf{2. Representation of National Income by Production Function
(class notes)}

\begin{itemize}
\item  Production function, technological change; potential $vs$ Actual
Output; trend in potential and actual output and business cycles

\item  Measurement of Growth (Linear and exponential growth rates;
Computation of Catching up time and Double up time; Growth Rates of
functions; Computation of growth Rates)
\end{itemize}

\item  \textbf{3. Accounting for Observed Growth (class notes, and articles)}

\begin{itemize}
\item  Capital: \emph{household savings, government budget deficit and
foreign capital flows }

\item  Labor: \emph{population, hours of work, and education, health status}

\item  Knowledge and productivity growth: $R\&D$\emph{, technology import}, 
\emph{direct foreign investment}

\item  \textbf{Macro Evidence on Sources of Growth}

\begin{itemize}
\item  U.S., Korea and Japan (Class notes)

\item  *An international Comparison: Korea, Japan, U.S., Canada, Belgium,
Denmark, France, West Germany, Italy, Norway, U.K. (Denison, 1962).
\end{itemize}
\end{itemize}

\item  \textbf{4. Basic Descriptive Models of Growth and Development
(lecture notes, and partly from Jones, Todaro, Ray)}

\begin{itemize}
\item  Stylized Facts about a few macro variables: Kaldor's findings on U.S.

\item  Harrod-Domar Model: no factor substitution and Knife-edge instability

\item  Solow model: factor substitutions and stability

\item  Solow model with Harrod-neutral technological change

\item  Lewis Dual Economy Growth Model (Todaro, pp:74-80; Ray, pp:x-x )
\end{itemize}
\end{description}

\textbf{PART II: Markets, Institutions, and Economic Development}

Our focus in this section will be on human resources. 

\begin{description}
\item  \textbf{5. Details on the components of labor (household decisions
and labor market outcomes)}

\begin{itemize}
\item  Distribution of earnings and its determinants

\begin{itemize}
\item  Labor hours

\item  Education

\item  Health -- malaria and other diseases leading to high infant mortality
and lower health status and living standard

\item  Fertility choices
\end{itemize}

\item  Labor mobility

\begin{itemize}
\item  Geographical mobility

\item  Social mobility
\end{itemize}

\item  Evaluation of social programs

\begin{itemize}
\item  an introduction to redistributive justice, cost-benefit analysis,
cost-effectiveness analysis, cost-utility analysis, and willingness-to-pay.
We will illustrate these for health and education.
\end{itemize}
\end{itemize}
\end{description}

\textbf{PART III: INTERNATIONAL ISSUES IN DEVELOPMENT}

\begin{description}
\item  \textbf{6. International Trade and Factor Movements and Development:}
(selected chapters from Ethier book).

\begin{itemize}
\item  International Trade (Basic models of pure trade, and implications for
development experiences of countries and trade policies)

\item  International labor movement (briefly)

\item  International Capital Flows
\end{itemize}

\item  \textbf{7. Financial Sector}

\begin{itemize}
\item  Nature of credit institutions in less developed countries; nature of
credit contracts in formal and informal lendings.

\item  Asian currency crisis (references will be provided)
\end{itemize}
\end{description}

\textbf{PART IV:}

\begin{description}
\item  \textbf{8. Endogenous growth Models (}endogenizing the engines of
growth, will be covered if time permits\textbf{) }
\end{description}

Out of many we consider the following ones

%\begin{description}
\begin{itemize}
\item  Intellectual property rights, patents, R\&D and technological change
-- Romer Model (see Jones)

\item  Human capital investment -- Lucas Model

\item  Population density and technological change (cf. Raut and Srinivasan)
\end{itemize}
%\end{description}

\end{document}
